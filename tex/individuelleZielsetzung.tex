
%\section{Ziele}\label{sec:2.1}
%Es soll ein 2.1-System entwickelt werden.
%Ein Mono-Subwoofer übernimmt den niedrigen Audiofrequenzbereich, während 2 weitere Lautsprecherboxen (auch genannt Satelliten-Boxen) - mit jeweils einem Hochton- und Tiefton-Lautsprecher - den Rest übernehmen.
%Dieser Aufbau wurde gewählt um einen Raum-Klang erzeugen zu können. \\
%Die Versorgung der Elektronik erfolgt entweder über Akku- oder Netzbetrieb.
%Jeder Hochton- und Tiefton-Lautsprecher bekommt einen eigenen Verstärker (so auch der Mono-Subwoofer). \\
%Das Audiosignal wird in verschiedene Frequenzbereiche aufgeteilt.\\
%
%%An Struktur neu anpassen
%Unsere Diplomarbeit kann grob in 2 Teile aufgeteilt werden:
%\begin{itemize}
%	\item Entwicklung der Elektronik
%	\item Auswahl und Messungen der Lautsprecher
%\end{itemize}
%Die Ziele der 2 Teile werden in diesem Kapitel kurz erläutert.


\section{Elektronik}\label{subsec:2.1.1}
Wie bereits in der Einleitung (\ref{sec:1.2}) erwähnt, ist das Projekt auch für den Akkubetrieb ausgelegt und benötigt daher eine passende Versorgungsschaltung.
Das Versorgungskonzept sieht einen 12V-Akku mit entsprechendem Ladegerät vor.
Dieses wurde zugekauft, da unser Projekt sich auf die Lautsprechermessung und -beschaltung konzentriert.
Bei Anschluss an eine Steckdose (nach CEE-Norm) übernimmt das vorgesehene Netzteil die Versorgung der Elektronik.
Wegen einer höheren Versorgungsspannung als die 12V des vorgesehenen Akkus, steht bei Netzbetrieb eine höhere Leistung zur Verfügung. \\ \\
Es werden analoge Verstärker verwendet.
Gründe dafür sind:
\begin{itemize}
	\item Einfacher Aufbau
	\item Ausreichende Leistung und Effizienz für dieses Projekt
	\item Bewährte Technik für Audioverstärker
\end{itemize} 
Das Audio-Signal muss vor den Verstärkern noch gefiltert werden.
Diese Aufgabe übernimmt eine aktive Frequenzweiche.
Für den Mono-Subwoofer wird eine eigene Schaltung entwickelt die nicht nur die unerwünschten Signal-Anteile herausfiltert, sondern auch noch das Stereo-Signal zu einem Mono-Signal addiert.\\ \\
%\newpage ??
Die Signalaufnahme erfolgt über Eingänge auf der Hauptplatine.
Auf dieser Platine befindet sich ein AUX-Eingang (3,5mm Klinkenbuchse).
Über ein passendes Kinkensteckerkabel kann das Audiosignal von einem beliebigen Endgerät wiedergegeben werden.
Eine weitere Möglichkeit bietet das Bluetooth-Modul, welches sich ebenfalls, mittels Adapterplatine, auf der Hauptplatine befindet.
Durch dieses kann über Bluetooth-Funkverbindung mit einem beliebigen bluetoothfähigen Endgerät ebenso ein Audiosignal übertragen werden.
An der Hauptplatine werden des weiteren die Weichen und an diesen die Verstärker angeschlossen. 
Das schlussendlich verstärkte und gefilterte Audiosignal wird an den Lautsprechern abstrahlen.\\
Die Hauptplatine und die Adapterplatine für das Bluetooth-Modul, wurden während einer Projektarbeit in der HTBLuVA St. Pölten entwickelt.
%Addierschaltung von Klinke und Bluetooth vielleicht erst später einbauen und erläutern

\section{Lautsprecher}\label{subsec:2.1.2}
Das Ziel für die Auswahl der Lautsprecher war es, möglichst laute spielende Lautsprecher-Chassis mit möglichst gutem Frequenzgang (großer verwendbarem Frequenzbereich im Zusammenhang mit geringer Welligkeit [Siehe Kapitel \ref{sec:8.6}]) zu finden.
Ein weiteres Ziel ist die Verringerung des Volumens der Box.
Dabei muss beachtet werden, dass sich die Klangqualität des in der Box verbauten Lautsprecher-Chassis mit ändernden Volumen, ebenfalls verändern kann.
Daher ist ein Boxenvolumen zu bestimmen unter diesem die Klangqualität des Lautsprecher-Chassis nicht leidet.\\
Ein kleines Volumen ist nötig, da das fertige Lautsprecher-System auch für den Außenbereich verwendet werden soll und daher leicht transportabel sein sollte.
Aus diesen Bedingungen folgt eine Optimierungsarbeit am Boxenvolumen.