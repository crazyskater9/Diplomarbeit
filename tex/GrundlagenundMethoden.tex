\section{Blockschaltbild des Projekts}\label{sec:3.1}
Nach einigen Überlegungen über die beste Herangehensweise an dieses Projekt, wurde folgendes Blockschaltbild entwickelt:
\begin{figure} [H]
	\centering
	\includegraphics[width=1\textwidth]{img/blockschaltbild.png}
	\caption{Blockschaltbild}
	\label{fig:3.1.1}
\end{figure}
Ein wichtiger Teil des Projekts ist die Bluetooth-Hauptplatine (in Abb. \ref{fig:3.1.1} links unten).
Diese Platine verfügt über die Audio-Eingänge des Projekts.
Das Stereo-Audio-Signal kann mithilfe der BT-Hauptplatine mittels Klinkenbuchse oder Übertragung über Bluetooth (siehe: Kap. \ref{sec:3.3}) in die gesamte Schaltung eingespeist werden.
\\ \\
Dieses Stereo-Audio-Signal wird dann mithilfe der Frequenzweichen in 3 verschiedene Frequenzbereiche (Hochton-, Mittel-Tiefton- und Bassbereich) aufgetrennt.
\\
Tiefe Frequenzen (<150Hz) werden \enquote{L + R} (Stereo) addiert und über eine Mono-Tiefpass-Weiche gefiltert.
\\
Insgesamt gibt es dann also 5 Signale, die weiterverarbeitet werden:
\\ 
\begin{itemize}
	\item Hochton-Links
	\item Hochton-Rechts
	\item Mittel-Tiefton-Links
	\item Mittel-Tiefton-Rechts
	\item Mono-Bassbereich
\end{itemize}

Die verschiedenen Audio-Signale werden dann mit je einem Leistungs-Verstärker verstärkt.
Die Ausgangsleistung der Leistungs-Verstärker variiert je nach Frequenzbereichen.
Dies ist bedingt durch die verwendete Verstärkerschaltung.\\
Diese Schaltungen sind für:
\begin{itemize}
	\item \textbf{Hochtonbereich:}\\ TDA2030-Leistungsverstärker-Grundschaltung
	\item \textbf{Mittel-Tieftonbereich:}\\ TDA2030-Leistungsverstärker mit Leistungstransistoren
	\item \textbf{Bassbereich:}\\ TDA2030-Leistungsverstärker mit Leistungstransistoren in H-Brücke
\end{itemize}
Aus diesen Schaltungen erschließen sich folgenden maximale Ausgangsleistungen, bei asymmetrischer Spannungsversorgung von 24V und 1\% Klirrfaktor:\\
\begin{itemize}
	\item Hochton-Leistungsverstärker (an 8 Ohm): < 6W
	\item Mittel-Tiefton-Leistungsverstärker (an 4 Ohm): < 11W
	\item Bassbereich-Leistungsverstärker (an 4 Ohm): < 44W
\end{itemize}

Versorgt wird die Elektronik mit unter Akkubetrieb mit 12 V oder 24 V unter Netzbetrieb, genauere Erklärung im Kapitel \ref{sec:3.4}.
Somit befindet man sich an der Untergrenze der Leistungs-Verstärker-Schaltungen, diese sind maximal bis \enquote{+/- 22V} oder \enquote{+ 44V asymmetrisch} ausgelegt.
%Da eine Blei-Vlies-Batterie für das Projekt vorgesehen war und diese zumeist 12V liefern, ist die Entscheidung von einer 12V Spannungsversorgung sehr nahe.
%Andere Alternativen wie zB. LiPo-Akkus könnten mehr Spannung und Strom liefern, diese sind jedoch um einiges vorsichtiger zu Behandeln (sehr hohe Energiedichte) und daher auch umständlicher (Lade-Überwachung, Brandgefahr).

\newpage
\section{Mehrweg-Lautsprechersysteme}\label{sec:3.2}
Ein Lautsprecher ist ein Bauelement, das ein elektrisches Signal in ein akustisches Signal (20 Hz bis 20 kHz) umwandelt.
Dabei wird eine Membran in Schwingung versetzt, die wiederum die umgebende Luft zum Schwingen bringt und einen hörbaren Ton erzeugt.
\\ \\
Nun wird aber nicht jede Frequenz gleich gut von ein und demselben Lautsprecher abgestrahlt.
Durch die physikalischen Gegebenheiten strahlen große Membranen tiefe Frequenzen besser ab als hohe Frequenzen.
Daher ist es sinnvoll, mehrere Lautsprecher für verschiedene Frequenzbereiche zu verwenden.
Benannt werden diese verschiedenen Lautsprecher durch den Bereich in dem sie am besten funktionieren.
Beispielsweise Hochton-Lautsprecher oder Hochtöner für den Hochton-Bereich (>2,5 kHz).
Wie gut nun verschiedene Frequenzen von einem Lautsprecher abgestrahlt werden, ist in seinem Frequenzgang ersichtlich:
\begin{figure} [H]
	\centering
	\includegraphics[width=1\textwidth]{img/LSMessung/TT1_9,17l_bestes.png}
	\caption{Beispiel eines Frequenzganges (Tieftöner PSS 297 58206)}
	\label{fig:3.2.1}
\end{figure}
Bei einem perfekten Lautsprecher würde diese Linie (Abb. \ref{fig:3.2.1}) eine parallele Gerade zur X-Achse bilden.
Da so ein Lautsprecher aber nicht existiert, werden verschiedene Frequenzen auch mit einem verschiedenen Schalldruckpegel vom Lautsprecher abgestrahlt.
\\
In dem Beispiel (Abb. \ref{fig:3.2.1}) wurde ein Tieftöner gemessen.
Das ist klar ersichtlich, da der Schalldruckpegel ab 5 kHz fast kontinuierlich fällt.
Tiefere Frequenzen werden, in einem gewissen Bereich, gleich wiedergegeben.
\\ \\
Je nach Frequenzbereich kann ein Ton vom menschlichen Ohr auch lokalisiert \mbox{werden}.
Hohe Frequenzen (<20 kHz) sind besser lokalisierbar als tiefe Frequenzen.
Bei sehr niedrigen Frequenzen (<150 Hz) ist der Ton gar nicht mehr lokalisierbar.
Diesen \mbox{Effekt} kann man nutzen und damit akustische Effekte erzeugen.
Einige Lautsprecher-Systeme, die diesen Effekt nutzen sind:
\begin{itemize}
	\item 2.0-System
	\item 2.1-System
	\item 5.1-System
\end{itemize}
Dabei steht die erste Zahl für die Anzahl der verteilten Lautsprecher und die zweite Zahl für die Anzahl der verwendeten Subwoofer.
\\
Je mehr verteilte Lautsprecher benutzt werden, desto bessere Raumklang-Effekte sind realisierbar.
Dadurch wird die Beschaltung aber auch komplizierter, da 5 verschiedene Audio-Signale verwendet werden.
\\ \\
Der Subwoofer ist ein spezieller Lautsprecher.
Er arbeitet im Bass-Bereich(20-150 Hz) und verarbeitet nur ein einziges Signal.
Diese Frequenzen sind nicht lokalisierbar, weshalb auch nur ein Subwoofer benötigt wird.
Meistens ist die Membran des Subwoofers um einiges größer als die Membran der anderen Lautsprecher.
Damit kann der Subwoofer mehr Luft in Bewegung setzen und somit einen höheren Schalldruckpegel erzeugen.
Um das zu ermöglichen benötigt der Subwoofer aber auch mehr Leistung als die anderen Lautsprecher.
\\ \\
In unserem Projekt haben wir uns direkt für ein 2.1-System entschieden, da sich dieses System am besten bewährt hat und für Musik ein Stereo-System völlig ausreicht.
Die zwei verteilten Boxen, auch Satelliten genannt, sind mit je einem Hochton- und einem Tiefton-Lautsprecher ausgestattet und werden über Kabeln mit der Hauptbox verbunden.
Damit sollen die Satelliten fast den gesamten Frequenzbereich für Audio (20 Hz bis 20 kHz) abdecken.
Der einzelne Subwoofer übernimmt den Bass-Bereich (<150 Hz) und sitzt in der Hauptbox.
Dort ist auch die gesamte Elektronik verbaut.

\section{Signalübertragung über Bluetooth}\label{sec:3.3}
Bluetooth ist eine moderne Funkschnittstelle für verschiedenste Anwendungen.
Unter anderem gibt es auch speziell für Audio-Anwendungen konzipierte Protokolle.
Die Übertragung läuft folgendermaßen ab:
\\
Zuerst muss das sendende Gerät (z.B. ein Smartphone) mit dem empfangenden Gerät (z.B. Bluetooth-Modul) verbunden werden.
Danach werden die gewünschten Daten ausgewählt.
In diesem Fall wären die Daten ein Musikstück.
Über Funk werden die digitalen Daten an das empfangende Gerät gesendet.
Nun muss das Bluetooth-Modul diese digitalen Daten wieder in ein analoges Signal umwandeln, welches dann weiterverarbeitet werden kann.
\\ \\
Dabei ist eine hohe Kompatibilität mit viele Geräten wichtig, weil es sehr viele verschiedene Versionen von Bluetooth gibt.
Da Bluetooth-Geräte meist abwärtskompatibel sind, ist es sinnvoll das Modul mit einer älteren BT-Version laufen zu lassen.
\\ \\
Nach ausführlicher Recherche wurde das Modul \enquote{XS3868} ausgewählt.
Der darauf verbaute Chip \enquote{OVC3860} von \enquote{OmniVision Technologies} hat sich bereits in vielen anderen Projekten bewährt, da er günstig ist und Funktionen wie \enquote{Play/Pause} bereitstellt.
\newpage
Im \enquote{OVC3860} (Abb. \ref {fig:3.3.1}) ist außer der Bluetooth-Verbindung auch noch ein Stereo-Audio-Prozessor verbaut.
Zusätzlich gibt es noch eine UART-Schnittstelle mithilfe man einige Einstellungen am Chip vornehmen kann.
Eine LiPo-Akku-Ladeschaltung ist ebenfalls vorhanden, wird aber in diesem Projekt nicht verwendet.
\\
Das Modul benötigt eine Versorgungsspannung von 3,3 V bis 4,2 V, wobei der Chip mit 1,8 V versorgt wird.
Diese Spannung (1,8 V) wird auf dem Modul erzeugt.
\\
Die verwendete BT-Version ist 2.0.
Einige GPIO-Pins sind auf das Modul herausgeführt um Funktionen wie \enquote{Play/Pause} zu ermöglichen.
Der Chip benötigt einen externen Speicher und eine Antenne (auf dem Modul) um ordnungsgemäß zu funktionieren.
\begin{figure} [H]
	\centering
	\includegraphics[width=1\textwidth]{img/BTModul/blockschaltbild.png}
	\caption[Blockschaltbild OVC3860]{Blockschaltbild OVC3860\footnotemark}\label {fig:3.3.1}
\end{figure}
\footnotetext{http://cxem.net/review/files/review24\_OVC3860.pdf,\\Zugriff: 11.03.2017}

\newpage
\section{Spannungsversorgung}\label{sec:3.4}
Die gesamte Box soll portabel sein, d.h. ohne externe Stromzufuhr funktionieren.
Dafür ist ein Akku notwendig.
Nach einigen Überlegungen haben wir uns für einen Blei-Vlies-Akku entschieden.
Gründe dafür sind:
\begin{itemize}
	\item Geringe Kosten
	\item Einfache Beschaltung
	\item Geringere Gefahr gegenüber Lithium-Akkus
	\item Wegen Vlies-Technik: Kein Auslaufen von chemischen Substanzen
\end{itemize}
Dieser Akku versorgt grundsätzlich die Elektronik mit 12 V und wird über ein passendes Ladegerät aufgeladen.
Da aber mit dieser, relativ geringen, Spannung nur eher kleine Leistungen zu erwarten sind, kam die Idee auf, die Verstärker bei externer Versorgung durch das Stromnetz (230 V / AC) mit einer größeren Spannung (24 V) zu versorgen.
Um das zu realisieren wird ein Netzteil, in unserem Fall ein Schaltnetzteil benötigt.
Falls das Gerät am externen Stromnetz hängt, wird die Versorgung der Verstärker automatisch mithilfe eines passenden Relais umgeschaltet.
Diese Lösung wurde gewählt, da sie sehr simpel ist und gut funktioniert.
Veranschaulicht wird das Konzept durch diese Schaltung:
\begin{figure} [H]
	\centering	
	\includegraphics[width=1\textwidth]{img/Grundlagen/Versorgung.png}
	\caption{Versorgungskonzept}
	\label {fig:3.4.1}
\end{figure}
Das Lautsprecher-System kann somit bei Netzbetrieb die Musik lauter abspielen, als bei Akku-Betrieb.

% ANDERE VERSION
%Da das Projekt portabel sein soll, ist auch ein Akku (Vlies-Blei-Akku) eingebaut.
%Er versorgt die gesamte Elektronik mit 12 V bei Akkubetrieb.
%Aufgeladen wird er durch ein passendes Ladegerät, welches aber nicht von uns entwickelt wird.
%\\ \\
%Falls eine externe Stromversorgung über eine Steckdose (230V AC) gegeben ist, ändert sich die Versorgung der Verstärker.
%Während nun der Akku aufgeladen wird, schaltet ein Relais die Versorgungsspannung der Verstärker von 12 V auf 24 V um.
%Diese Spannung (24 V) wird durch ein Schaltnetzteil erzeugt.
%\\ \\
%Durch eine höhere Spannung können die Verstärker auch die Signale auf höhere Spannungen verstärken.
%Somit ergibt sich eine höhere Leistung an den Verstärkern aber auch an den Lautsprechern.
%Eine Leistungssteigerung an einem Lautsprecher entspricht einer Steigerung des Schalldruckpegels was wiederum eine höhere Lautstärke bewirkt.
%An den Verstärkern bewirkt eine höhere Leistung auch eine größere Wärmeentwicklung.
%Der verbaute Kühlkörper muss diese Wärme bei Akkubetrieb als auch bei Netzbetrieb ableiten können.
%\\ \\
%Das Lautsprecher-System kann somit bei Netzbetrieb die Musik lauter abspielen, als bei Akku-Betrieb.
