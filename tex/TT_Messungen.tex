%TD -- Kapitel 4.2 -- alle Labels darauf aufbauend

%\newpage
\section{Lautsprecher für Tiefton-Bereich} \label{sec:4.2}
\subsection*{Einleitung} \label{subsec:4.2.1}
Tiefton-Lautsprecher sind, wie der Name bereits andeutet, für den unteren Frequenzbereich gedacht und konzipiert.
In diesem Projekt werden Tiefton-Lautsprecher an zwei verschiedenen Positionen verwendet:
\begin{itemize}
	\item im Subwoofer und 
	\item in den Satellitenboxen
\end{itemize}
Es ist so angedacht, dass der Subwoofer (Mono) nur die sehr niedrigen Frequenzen übernimmt.
Er ist bis ca. 150 Hz aktiv.\\
In den Satellitenboxen ist jeweils ein kleinerer Tiefton-Lautsprecher verbaut, um den Übergang zwischen sehr tiefen und hohen Frequenzen zu bilden.
Diese zwei Lautsprecher (Stereo) arbeiten ebenfalls bei niedrigen als auch bei mittleren bis hohen Frequenzen.
Die Grenze des Satelliten-Tiefton-Lautsprechers liegt bei ca. 5 kHz.
Somit wird die Frequenzweiche für diese Frequenz berechnet und konzipiert.

\subsection*{Ziele} \label{subsec:4.2.2}
Der Lautsprecher für die Subwoofer-Box muss größer und leistungsfähiger sein, um die richtig tiefen Frequenzen möglichst gut abzustrahlen.
Da er nur bis ca. 150 Hz aktiv ist, sollte der Frequenzgang genau in diesem Bereich einen hohen Schalldruckpegel aufweisen.
Der Rest des Frequenzganges ist nicht so wichtig, da der Lautsprecher auch nicht in einem höheren Bereich verwendet wird. \\
Bei den Satelliten-Lautsprechern ist der ganz tiefe Frequenzbereich (< 150 Hz) nicht so ausschlaggebend.
Wichtiger ist dafür eine niedrige Welligkeit (Kap. \ref{sec:8.6}) bis 5 kHz, um den mittleren Frequenzbereich gut abstrahlen zu können.\\ \\
%Welligkeit könnte man vielleicht auch erklären als Grundlage -- @Bointii
Der Messvorgang und -aufbau wurde bereits im Kapitel \ref{sec:4.1} erläutert.
Nach diesem Prinzip wird bei allen Messungen vorgegangen.
%Nach diesem Prinzip wird bei fast allen Messungen vorgegangen.

\newpage
\subsection*{Subwoofer} \label{subsec:4.2.3}
Bereits zu Beginn der Diplomarbeit wurde ein großer Tiefton-Lautsprecher eingekauft.\\
Dieser wurde in verschiedenen Volumina gemessen.\\
Außerdem wurden Vergleichsmessungen mit einem zweiten Tiefton-Lautsprecher durchgeführt.
\begin{figure} [H]
	\centering
	\includegraphics[width=0.4\textwidth]{img/LSMessung/TT/renkforce_B12123.png}
	\caption[\enquote{Renkforce B12123}]{\enquote{Renkforce B12123}\footnotemark}
	\label{fig:4.2.3.1}
\end{figure}
\footnotetext{https://www.conrad.at/de/auto-subwoofer-chassis-300-mm-500-w-renkforce-4-370335.html,\\Zugriff: 17.02.2017}
Dieser Lautsprecher wurde ausgewählt, weil er einen vergleichsweise hohen Schalldruckpegel bei tiefen Frequenzen aufweist und laut Angaben des Herstellers nur ein geringes Boxenvolumen benötigt.
Kurze Spezifikationen des Subwoofers sind:
\begin{itemize}
	\item Maximale Spitzenbelastbarkeit 500 W
	\item Maximale RMS-Belastbarkeit 200 W
	\item Durchmesser 30 cm
	\item Schalldruck 93 dB
	\item Impedanz 4 $\Omega$
\end{itemize}
Genauere Angaben findet man im Datenblatt des Lautsprechers (Kap. \ref{sec:8.7}).
Da man eine Box einfacher verkleinern kann als vergrößern, wurde eine Box mit einem Volumen von 149 l für die Messungen verwendet.
Die Box wurde aus Holz gefertigt und mit Silikon abgeschlossen, um mögliche Luftlöcher zu schließen.
Als Vergleichslautsprecher wurde ein \enquote{Visaton WPC30} gemessen.

\newpage
Diese Messungen wurden zu Beginn der Diplomarbeit durchgeführt und daher sind die Ergebnisse so aufgenommen, dass die Einstellungen der Software sichtbar sind. % Ideale Messung beschreiben?
Um die Aufnahmen ansehnlicher zu machen wurde der Frequenzgang durch Nachbearbeitung ausgeschnitten, sodass im Gegensatz zur Abbildung \ref{fig:4.1.1.1} in den nachfolgenden Bildern (ab Abb. \ref{fig:4.2.3.2}) nur die zum Vergleich relevanten Kurven dargestellt werden.\\
Die zwei Subwoofer-Chassis wurden einmal mit und einmal ohne Wolle in der Box gemessen.
\begin{figure} [H]
	\centering
	\subfloat{\includegraphics[width=1\textwidth]{img/LSMessung/TT/RenkforceOhneWolleMitSilikonV2.png}}\quad
	\subfloat{\includegraphics[width=1\textwidth]{img/LSMessung/TT/VisatonMitSilikonOhneWolleV2.png}}
	\caption{Subwoofer-Messung ohne Wolle\\ \enquote{Renkforce} (oben) und \enquote{Visaton} (unten) | Boxenvolumen: 149 l}
	\label{fig:4.2.3.2}
\end{figure}
Bei diesem Vergleich ist sichtbar, dass der Frequenzgang des \enquote{Visaton}-Tieftöners allgemein \enquote{gerader} ist und somit eine geringere Welligkeit aufweist.
Wenn man allerdings nur auf die tiefen Frequenzen (< 150 Hz) achtet, die auch verwendet werden, ist festzustellen, dass der Schalldruckpegel des \enquote{Renkforce} einen höheren Wert aufweist.
Da der Lautsprecher zusätzlich in diesem Bereich auch noch eine geringe Welligkeit aufweist, ist er zu bevorzugen.
\newpage
Eine weitere Messung mit Wolle in der Box wurde deshalb durchgeführt, weil beide Frequenzgänge bei ca. 300 Hz eine ähnliche Unregelmäßigkeit aufweisen.
Die Auswirkungen auf das Messergebnis sind folgende:
\begin{figure} [H]
	\centering
	\subfloat{\includegraphics[width=1\textwidth]{img/LSMessung/TT/RenkforceMitWolleMitSilikonV2.png}}\quad
	\subfloat{\includegraphics[width=1\textwidth]{img/LSMessung/TT/VisatonMitSilikonMitWolleV2.png}}
	\caption{Subwoofer-Messung mit Wolle\\ \enquote{Renkforce} (oben) und \enquote{Visaton} (unten) | Boxenvolumen: 149 l}
	\label{fig:4.2.3.3}
\end{figure}
Beide Frequenzgänge wurden durch das Hinzufügen von Wolle etwas \enquote{glatter}, also weisen weniger Welligkeit auf.
Die Wolle kann z.B. stehende Wellen in der Box verhindern und verbessert so die Eigenschaften der gesamten Box.\\ \\	%Es wirkt wie eine Vergrößerung des Volumens - hat Wagn einmal gsagt
Der Lautsprecher \enquote{Renkforce} weist insgesamt bei tieferen Frequenzen einen höheren Schalldruckpegel auf und wurde auch deshalb für dieses Projekt ausgewählt.

\newpage
\subsection*{Satelliten-Tiefton-Lautsprecher} \label{4.2.4}
Als Satelliten-Tiefton-Lautsprecher kommen mehrere Lautsprecher-Chassis in Frage, die vom Betreuer zur Verfügung gestellt wurden.
Die Lautsprecher-Chassis haben einen ähnlichen Durchmesser.
Dadurch können sie in einem Gehäuse (à 13,72 Liter) mit wechselbarer Frontplatte gemessen werden.
Um möglicherweise bessere Ergebnisse erzielen zu können, wurde das Volumen mittels Ton-Ziegelstein, Styropor oder Ytong-Ziegel verringert, oder durch Wolle vergrößert.\\
Da das Gehäuse zu Beginn nicht komplett luftdicht abgeschlossen war, wurde es an den offenen Stellen mit Silikon verschlossen.
Um bei den wechselnden Frontplatten einen luftdichten Verschluss zu ermöglichen, wurde ein Gummi zwischen Gehäuse und Frontplatte verbaut.
Die Frontplatten wurden mittels Schrauben in der vorgesehenen Öffnung befestigt.\\ \\
Zu den gemessenen Chassis gehören:
\begin{itemize}
	\item \enquote{TT1}: PSS 297 58206 100W 6Ohm
	\item \enquote{TT2}: SAMCO 10D1K06 20W 8Ohm
%	\item \enquote{TT3}: Infinity, ein Auto-Lautsprecher
\end{itemize}
Die Bezeichnung TT1 und TT2 dient zur Vereinfachung.\\

%\newpage
Wie zuvor erwähnt wurden die Messungen zu Beginn der Diplomarbeit über einen Screenshot festgehalten.
Man wurde im Verlauf der Diplomarbeit auf eine bessere Methode zur Dokumentation hingewiesen, welche im Laufe der Messungen angewandt wurde.
Jene Verbesserung wird auf zwei Punkte bezogen:\\
\begin{enumerate}
	\item Die Skalierung der Y-Achse sollte von 0 bis 60 dB reichen, da diese Skalierung normalerweise für Lautsprechermessungen verwendet und so dieser Bereich genauer aufgelöst wird.
	\item Es sollte nur das Fenster des Frequenzganges gespeichert werden, um eine höhere Auflösung für diesen zu erhalten.
\end{enumerate}

\newpage
\textit{Die folgenden Messungen wurden bereits mit dem mit Silikon verschlossenen Gehäuse erstellt.}\\
\begin{figure} [H]
	\centering
	\subfloat{\includegraphics[width=1\textwidth]{img/LSMessung/TT/TT1mitSilikonV2.png}}\quad
	\subfloat{\includegraphics[width=1\textwidth]{img/LSMessung/TT/TT1mitSilikonUndWolleV2.png}}
	\caption{Tiefton-Lautsprecher-Messung ohne Wolle\\ \enquote{TT1} (oben) und mit Wolle \enquote{TT1} (unten) | Boxenvolumen: 13,72l}
	\label{fig:4.2.4.1}
\end{figure}
Es ist die Verringerung der Welligkeit im Tieftonbereich (<1 kHz) sichtbar.
In dieser Situation sollte der Tiefton-Lautsprecher im besten Fall bis 1 kHz arbeiten.
Des weiteren sollte er besser mit Wolle versehen werden.

\newpage
Als Referenz wird der Lautsprecher \enquote{TT2} gemessen. 
Ebenfalls einmal ohne und einmal mit Wolle im Gehäuse.
\begin{figure} [H]
	\centering
	\subfloat{\includegraphics[width=1\textwidth]{img/LSMessung/TT/TT2mitSilikon_ohneWolleV2.png}}\quad
	\subfloat{\includegraphics[width=1\textwidth]{img/LSMessung/TT/TT2mitSilikon_mitWolleV2.png}}
	\caption{Tiefton-Lautsprecher-Messung ohne Wolle\\ \enquote{TT2} (oben) und mit Wolle \enquote{TT2} (unten) Boxenvolumen: 13,72 l}
	\label{fig:4.2.4.2}
\end{figure}
Dieser Lautsprecher weist eine einigermaßen vertretbare Welligkeit im Mittel- und Hochton-Bereich auf.
Beim Niederton-Bereich ist der Schalldruckpegel jedoch geringer.
Der Mono-Subwoofer müsste bis einige 100 Hz aktiv sein, um diese Schwäche auszumerzen.
Ein Subwoofer sollte jedoch nicht über 150 bis 200 Hz kommen, da Frequenzen unterhalb dieser Grenze nicht lokalisiert werden können und, dies der Sinn eines Subwoofers wäre.
Wieder ist ersichtlich, dass mit Wolle die Welligkeit im Mittel und Tiefton-Bereich etwas verringert wird.

\newpage
Durch ein paar weitere Messungen wurde der TT2 bereits sehr früh aus der Auswahl genommen und der Fokus auf den TT1 gelegt, da dieser im Tieftonbereich besser funktioniert.
\\
Außerhalb der Messungen für die Diplomarbeit wurde der Lautsprecher TT1 etwas beschädigt.
Die \enquote{Dustcap} des Lautsprechers wurde eingedrückt.
Für einen Tiefton-Lautsprecher ist das nicht so schlimm, da die \enquote{Dustcap} in diesem Fall rein für den Schutz vor Staub zuständig ist.
Dies ist notwendig, um den Bewegungsfreiraum zwischen der Spule an der beweglichen Membran und der Spule am fixen Gehäuse gewährleisten zu können.
Wenn dieser Schutz nicht vorhanden ist, ist meist auch der Lautsprecher kaputt.\\
Bei einem Hochton-Lautsprecher wäre bereits das Eindrücken der \enquote{Dustcap} ein Todesurteil für den Lautsprecher, da die Halbkugelstruktur der \enquote{Dustcap} in diesem Frequenzbereich einen immensen Einfluss auf den Klang nimmt.
\\ \\
Nach Feststellen des Schadens wurde der Frequenzgang des TT1 erneut überprüft und er wies keine wesentlichen Veränderungen im Spektrum auf, was zu dem Schluss führte, dass er normal weiterverwendet werden kann.
\\

% IN KAPITEL OPTIMIERUNG !
%Als nächster Schritt werden die Tieftöner unter variierendem Volumen gemessen.
%Dafür wird, wie zuvor erwähnt, hauptsächlich Ytong und Styropor verwendet.
%Styropor gilt auch als einigermaßen schalltotes Material.
%Da keine fixen Angaben gemacht wurden was die Schalldichte von Styropor betrifft, wurde eine Referenzmessung mit Ziegeln gemacht(Siehe Kapitel \ref{label}). % Referenzmessung Styropor, Ziegel einbauen!
%\\
%Das Volumen wird zuerst einmal stark verringert, um zu sehen ob überhaupt ein Effekt merkbar ist.
%Dafür wird das Volumen von 13,72l auf 4,87l verringert.
%\begin{figure} [H]
%	\centering
%	\includegraphics[width=0.7\textwidth]{img/LSMessung/TT/TT1silikon_4-87lVolumen.png}
%	\caption{\enquote{TT1} | Boxenvolumen = 4,87 l}
%	\label {fig:5.3.4.3}
%\end{figure}

%Vergleich mit original machen











