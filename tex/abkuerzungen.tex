%% start of file abkuerzungen.tex

% Abkuerzungsverzeichnis
\addchap{
	\iflanguage{english}{Acronyms}{Abkürzungsverzeichnis}}
\begin{acronym}[ACRONYM]
\acro{aux}[AUX-Input]{Auxilary Input - dt. Zusatz-Eingang}
%\acro{abb}[Abb.]{Abbildung}
\acro{bt}[BT]{Bluetooth}
\acro{cee}[CEE-Norm]{Commission on the Rules for the Approval of the Electrical Equipment - dt. internationale Kommission für die Regelung der Zulassung elektrischer Ausrüstungen}
%\acro{db}[dB]{Dezibel}
\acro{dek}[Dek.]{Dekade}
\acro{elko}[ELKO]{Elektrolyt-Kondensator}
\acro{emv}[EMV]{Elektro-Magnetische Verträglichkeit}
%\acro{f}[F]{Farat}
\acro{gpio}[GPIO]{General Purpose Input/Output - dt. Allzeckeingabe/-ausgabe}
\acro{hifi}[HiFi]{High Fidelity - dt. Hohe (Klang-)Treue}
\acro{ht}[HT]{Hochton-Lautsprecher}
\acro{HTBLuVA}[HTBLuVA]{Höhere Technische Bildungs, Lehr- und Versuchsanstalt}
%\acro{hz}[Hz]{Hertz}
\acro{ic}[IC]{Integrated Circuit - dt. Integrierter Schaltkreis}
%\acro{kap}[Kap.]{Kapitel}
\acro{led}[LED]{Light Emitting Diode - dt. Licht emittierende Diode}
\acro{Lipo}[LiPo]{Lithium Polimer}
\acro{opv}[OPV]{Operationsverstärker}
\acro{pcb}[PCB]{Printed Circuit Board - dt. Leiterplatte}
\acro{Rms}[RMS]{Rout-Mid-Square - dt. Effektivwert}
\acro{smd}[SMD]{Surface Mouted Device - dt. Auf der Oberfläche montierter Bauteil}
%%\acro{spi}[SPI]{Serial Peripheral Interface}
%%\acro{tikz}[TikZ]{\TikZ{} ist kein Zeichenprogramm}
\acro{tt}[TT]{Tiefton-Lautsprecher}
\acro{uart}[UART]{Universal Asyncron Reciev Transmitter - dt. Universelle Asynchrone Empfangs- und Übertragungsschnittstelle}
\acro{usb}[USB]{Universal Serial Bus - dt. Universaler Serieller Bus}
%\acro{v}[V]{Volt}
\acro{vbat}[VBAT]{Batterie Voltage - dt. Batteriespannung}
\acro{vcc}[Vcc]{Positive Versorgungsspannung}
\acro{via}[VIA]{Durchkontaktierungen einer Doppelseitigen-Leiterplate}
%\acro{zb}[zB.]{zum Beispiel}

% Hab mal das auskommentiert was meiner Meinung nach nicht unbedingt nötig ist ;)

\end{acronym}\newpage

%% end of file abkuerzungen.tex