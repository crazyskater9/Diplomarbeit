%% Dokumentklasse KOMA-Script Report
\documentclass[paper=a4, 12pt]{scrreprt}
%% Encoding UTF8
\usepackage[utf8]{inputenc}
%% 8 Bit Aufloesung der Buchstaben
\usepackage[T1]{fontenc}
%% Seitenraender
\usepackage[scale=0.72]{geometry}
%% Spracheinstellungen
\usepackage[english, naustrian]{babel} % your native language must be the last one!!
%% erweiterte Farbenpalette
\usepackage[dvipsnames]{xcolor}
%% Abbildungen
\usepackage{graphicx}
%% Tabellen (erweitert)
\usepackage{tabularx}
%% TikZ + Circuit-TikZ (fuer Schaltungen)
\usepackage[europeanresistors, europeaninductors]{circuitikz}
%% Nuetzliche TikZ Libraries
\usetikzlibrary{arrows, automata, positioning}
%% mathematik
\usepackage{amsmath, amssymb}
%\usepackage{mathtools}	
%% pdf-einbindung
\usepackage{pdfpages}
%% scource-code einbindung
\usepackage{listings, scrhack} %scrhack vermeidet Umschaltung auf KOMA Floats..
\usepackage{courier}
%% euro-symbol
\usepackage{eurosym}
%% landcsape-seiten ermöglichen
\usepackage{lscape}

%% Diplomarbeits-Format
\usepackage{srdpdipa}

%% Abkuerzungsverzeichnis
\usepackage[]{acronym}

%% Todos
\usepackage[]{todonotes}

%% Ganttdiagramme
\usepackage{pgfgantt}

%% Subfigures
\usepackage[lofdepth]{subfig}

%% definitionen =====================================%%
\dataSchool{HTBLuVA St. Pölten}
\dataDepartment{Höhere Lehranstalt für Elektronik und Technische Informatik}
\dataSubdepartment{Ausbildungsschwerpunkte Embedded- \& Wireless Systems}
\dataSession{2016/17}

\title{Bluetooth-Aktivbox}
\author{Markus Bointner \and Andreas Macsek}
\date{\today}
\place{St. P\"olten}
\professor{Prof. Dipl.-Ing. Dr. Herbert Wagner}
%%====================================================%%

% Hyperlinks im Dokument
\usepackage[colorlinks=true,
    linkcolor=black,
    citecolor=green,
    bookmarks=true,
    urlcolor=black,
    bookmarksopen=true]{hyperref}

\begin{document}

\frontmatter

%% titelseite ==========================================%%
\maketitle
%%======================================================%%

%% komplett leere seite ================================%%
\newpage\null\thispagestyle{empty}\newpage
%%======================================================%%

%% eidesstattliche erklärung ===========================%%
\begin{affidavit}
    \unterschrift{Markus Bointner}
    \unterschrift{Andreas Macsek}
\end{affidavit}
%%======================================================%%

%% dokumentation (deutsch/englisch) ====================%%
\includepdf[pages=-]{form/dokumentation-de.pdf}
\includepdf[pages=-]{form/dokumentation-en.pdf}
%%======================================================%%

%% inhaltsverzeichnis ==================================%%
\tableofcontents
%%======================================================%%

%% HAUPTTEIL ===========================================%%
\responsible{Markus Bointner, Andreas Macsek}
\mainmatter

\chapter{chapter}
\section{section}
\subsection{subsection}
\subsubsection{subsubsection}
\paragraph{paragraph}
blabla
\subparagraph{subparagraph}

\chapter{Einleitung}
	    Hier ist die Einleitung.

\chapter{Individuelle Zielsetzung}
    \section{HF-Teil}
        Dieser Teil wurde von Max Musterschüler entwickelt. das ist ein test von git
    
    \section{Filter}
        Dieser Teil wurde von Otto Normalschüler entwickelt.
        Test

\chapter{Grundlagen und Methoden}

\chapter{Ergebnisse}


%% ANHANG ==============================================%%
\appendix

%% abkürzungsverzeichnis ===============================%%
%% start of file abkuerzungen.tex

% Abkuerzungsverzeichnis
\addchap{
	\iflanguage{english}{Acronyms}{Abkürzungsverzeichnis}}
\begin{acronym}[ACRONYM]
\acro{aux}[AUX-Input]{Auxilary Input - dt. Zusatz-Eingang}
%\acro{abb}[Abb.]{Abbildung}
\acro{bt}[BT]{Bluetooth}
\acro{cee}[CEE-Norm]{Commission on the Rules for the Approval of the Electrical Equipment - dt. internationale Kommission für die Regelung der Zulassung elektrischer Ausrüstungen}
%\acro{db}[dB]{Dezibel}
\acro{dek}[Dek.]{Dekade}
\acro{elko}[ELKO]{Elektrolyt-Kondensator}
\acro{emv}[EMV]{Elektro-Magnetische Verträglichkeit}
%\acro{f}[F]{Farat}
\acro{gpio}[GPIO]{General Purpose Input/Output - dt. Allzeckeingabe/-ausgabe}
\acro{hifi}[HiFi]{High Fidelity - dt. Hohe (Klang-)Treue}
\acro{ht}[HT]{Hochton-Lautsprecher}
\acro{HTBLuVA}[HTBLuVA]{Höhere Technische Bildungs, Lehr- und Versuchsanstalt}
%\acro{hz}[Hz]{Hertz}
\acro{ic}[IC]{Integrated Circuit - dt. Integrierter Schaltkreis}
%\acro{kap}[Kap.]{Kapitel}
\acro{led}[LED]{Light Emitting Diode - dt. Licht emittierende Diode}
\acro{Lipo}[LiPo]{Lithium Polimer}
\acro{opv}[OPV]{Operationsverstärker}
\acro{pcb}[PCB]{Printed Circuit Board - dt. Leiterplatte}
\acro{Rms}[RMS]{Rout-Mid-Square - dt. Effektivwert}
\acro{smd}[SMD]{Surface Mouted Device - dt. Auf der Oberfläche montierter Bauteil}
%%\acro{spi}[SPI]{Serial Peripheral Interface}
%%\acro{tikz}[TikZ]{\TikZ{} ist kein Zeichenprogramm}
\acro{tt}[TT]{Tiefton-Lautsprecher}
\acro{uart}[UART]{Universal Asyncron Reciev Transmitter - dt. Universelle Asynchrone Empfangs- und Übertragungsschnittstelle}
\acro{usb}[USB]{Universal Serial Bus - dt. Universaler Serieller Bus}
%\acro{v}[V]{Volt}
\acro{vbat}[VBAT]{Batterie Voltage - dt. Batteriespannung}
\acro{vcc}[Vcc]{Positive Versorgungsspannung}
\acro{via}[VIA]{Durchkontaktierungen einer Doppelseitigen-Leiterplate}
%\acro{zb}[zB.]{zum Beispiel}

% Hab mal das auskommentiert was meiner Meinung nach nicht unbedingt nötig ist ;)

\end{acronym}\newpage

%% end of file abkuerzungen.tex
%%======================================================%%

%% abbildungsverzeichnis ===============================%%
\setcounter{lofdepth}{2}
\dipalistoffigures
%%======================================================%%

%% tabellenverzeichnis =================================%%
\setcounter{lotdepth}{2}
\dipalistoftables
%%======================================================%%

%% danksagungen=========================================%%
\newpage
\begin{acknowledgements}
    Wir bedanken uns bei
    \subparagraph{Prof. Dipl.-Ing. Dr. Herbert Wagner} für die tatkräftige Unterstützung und das ermöglichen eines einwandfreien Arbeitens an unserer Diplomarbeit und dem Projekt.
    %\subparagraph{FL Ing. DEFG} für ...
\end{acknowledgements}

%%======================================================%%

%% literaturverzeichnis ================================%%
\newpage
\begin{literature}
% The TeXbook by D. E. Knuth
%\bibitem[1]{TeXbooooook}{\textbf{Donald~E.~Knuth:} \emph{The \TeX{}book}. 1986, {\scshape Addison--Wesley} Verlag,\\ ISBN-13: 978-0-201-13447-6}
\bibitem[1]{TDA2030Amp}{\textbf{Herbert Sax:} SGS-Ates TDA2030 , {\scshape SGS-Ates} Verlag,\\ ISBN-13: -}

\end{literature}
 
%%======================================================%%

%% betreuungsprotokolle ================================%%
\includepdf[pages=-]{form/betreuungsprotokoll_1.pdf}
\includepdf[pages=-]{form/betreuungsprotokoll_2.pdf}
%% =====================================================%%

\end{document}
